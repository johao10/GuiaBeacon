\documentclass[12pt]{article}
\usepackage[T1]{fontenc}
\usepackage[utf8]{inputenc}
\usepackage[spanish]{babel}

\begin{document}

\setlength{\parindent}{0cm} %quita la pequeña linea de inicio en todo el trabajo
%\newpage salto de pagina
%\parindent quitar sangria
\section{PROBLEMA}

\noindent La ingeniería es definida como el conjunto de conocimientos científicos y tecnológicos para la innovación, invención, desarrollo y mejora de técnicas y herramientas para satisfacer las necesidades y resolver los problemas de las empresas y la sociedad, por esto se aborda en este proyecto uno de los problemas más comunes pero poco visibles dentro de la Torre Tecnológica, que es el desconocimiento de las actividades que se llevan a cabo en los espacios de esta, una gran parte de los estudiantes desconoce los horarios, no ubica con facilidad algunos salones, laboratorios y otros ambientes debido distribución de espacios los horarios originales se modifican a lo largo del semestre, o los profesores deciden a última hora desplazarse a otras ubicaciones para efectuar las labores académicas. 
hola mundio

\section{PROPUESTA}
\noindent Este proyecto tiene como finalidad apoyar la difusión de información confiable relacionada con las actividades que se llevan a cabo en cada área o espacio físico de la universidad. Por esto surge la idea de aplicar la tecnología bluetooth con un dispositivo que está teniendo mucho éxito en mercadeo y publicidad, el Beacon. Este pequeño aparato permite proporcionar información a las personas a través de sus smartphones, basada en la ubicación exacta dada por el reconocimiento del beacon. Entre las funcionalidades del modelo tenemos la ubicación de los diversos ambientes clasificados por su tipo como entretenimiento, laboratorios, etc. Difusión de información de cada ambiente, y la visualización de mapas por piso

\newpage

\section{DESCRIPCIÓN DEL MODELO}

\noindent El modelo busca interactuar con el usuario utilizando tecnología beacon el cual es un sistema de posicionamiento en interiores (de sus siglas en inglés IPS, indoor position system) con los que mediante transmisores de bajo coste y consumo se pueden obtener datos de proximidad muy precisos. En vez de usar satélites o antenas se utilizan dispositivos similares pero preparados para interiores con tecnologías: WIFI, Bluetooth o incluso que funcionan por acústica u óptica.

Se tendrá en cuenta el uso de 1 dispositivo beacon por ambiente, a lo que nos lleva que mayor numero de beacon mayor es la proximidad de ubicación. Se registrará a los usuarios de la Universidad para una visualización exclusiva de los ambientes.

Entre las funcionalidades del modelo tenemos la ubicación de los diversos ambientes clasificados por su tipo como entretenimiento, laboratorios, etc. Difusión de información de cada ambiente, y la visualización de mapas por piso




\end{document}